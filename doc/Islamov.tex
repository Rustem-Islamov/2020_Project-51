\documentclass[12pt, twoside]{article}
\usepackage{jmlda}
\newcommand{\hdir}{.}
\usepackage[utf8]{inputenc}
\usepackage[english,russian]{babel}
\usepackage{graphicx}
\newcommand{\real}{\mathbb{R}}
\newcommand{\nat}{\mathbb{N}}
\newcommand{\integ}{\mathbb{Z}}
\usepackage{bm}


\begin{document}

\title
    [Анализ свойств ансамбля локально аппроксимирующих моделей] % краткое название; не нужно, если полное название влезает в~колонтитул
    {Анализ свойств ансамбля локально аппроксимирующих моделей}
\author
    [Р.\,И.~Исламов, А.\,В.~Грабовой, В.\,В.~Стрижов] % список авторов (не более трех) для колонтитула; не нужен, если основной список влезает в колонтитул
    {Р.\,И.~Исламов, А.\,В.~Грабовой, В.\,В.~Стрижов} % основной список авторов, выводимый в оглавление
    [Р.\,И.~Исламов$^1$, А.\,В.~Грабовой$^1$, В.\,В.~Стрижов$^{1}$] % список авторов, выводимый в заголовок; не нужен, если он не отличается от основного
\email
    {islamov.ri@phystech.edu; grabovoy.av@phystech.edu;  strijov@ccas.ru}
%\thanks
%    {Работа выполнена при
%     %частичной
%     финансовой поддержке РФФИ, проекты \No\ \No 00-00-00000 и 00-00-00001.}
\organization
    {$^1$Московский физико-технический институт}
\abstract
    {Данная работа посвящена анализу свойств ансамбля локальных моделей. Для задачи регрессии  предлагается использовать многоуровневый подход, согласно которому множество объектов разбивается на несколько подмножеств и каждому подмножеству соответствует одна локальная модель. Рассматривается задача построения универсального аппроксиматора --- мультимодели, которая представлена в виде совокупности локальных моделей. В качестве решающей функции используется выпуклая комбинация локальных моделей.  Коэффициенты выпуклой комбинации --- шлюзовая функция --- функция, значение которой зависит от объекта, для которого производится предсказание. Такой подход позволяет описывать те выборки, которые затруднительно описывать одной моделью. Для анализа свойств проводится вычислительный эксперимент. В качестве данных используются синтетические и реальные выборки. В данной работе реальные данные представлены выборками из boston house prices dataset, servo dataset.  
	
\bigskip
\noindent
\textbf{Ключевые слова}: \emph {локальная модель; линейные модели; ансамбль моделей.}
}

\maketitle
\linenumbers

\section{Введение}
В данной работе исследуется проблема построения мультимодели --- ансамбля локальных моделей. \textit{Локальная модель} --- модель, которая обрабатывает объекты, находящиеся в определенной связной области в пространстве объектов. В качестве агрегирующей функции используется выпуклая комбинация локальных моделей, при этом веса локальных моделей не постоянны, а зависят от положения объекта в пространстве объектов. 

Подход к мультимоделированию предполагает, что вклад каждой локальной модели в ответ зависит от рассматриваемого объекта. Мультимодель использует шлюзовую функцию, которая определяет значимость предсказания каждой локальной модели, входящей в ансамбль.

В данной работе каждая локальная модель является линейной. В качестве функционала качества рассматривается логарифм правдоподобия модели. Предлагается алгоритм нахождения оптимальных параметров ансамбля и локальных моделей. 

Преимуществом данного подхода является его способность описывать те выборки, которые затруднительно описывать одной моделью, и разбивать выборку в соответствии с выбранными моделями.

Алгоритмы тестировались на синтетических и реальных данных. Реальные данные представляли собой boston house prices и servo datasets. Эксперименты показали преимущество использования многоуровневой модели и смеси моделей по сравнению с использованием одной модели.


В прикладных задачах данные порождены в результате использования нескольких источников, либо гипотеза порождения и вовсе не известна. В таких случаях качество предсказания можно повышать увеличивая количество моделей. Если моделей на самом деле меньше, чем предполагается, то веса лишних моделей будут малы и их вклад
будет несущественен. Этим объясняется актуальность использования мультимоделирования.

\subsection{Работы по теме}

С момента своего появления мультимодельный подход стал предметом многих исследований. Были предложены различные типы архитектур локальных моделей, такие как SVM \cite{Collobert2002}, Гауссовский процесс \cite{Tresp01mixturesof}  и нейронные сети \cite{Shazeer2017}. Другие работы была сосредоточены на различных конфигурациях, таких как иерархическая структура \cite{NIPS1991_514}, бесконечное число экспертов \cite{Rasmussen} и последовательное добавление экспертов \cite{Aljundi2016}. \cite{garmash-monz-2016-ensemble} предлагает модель ансамбля локальных моделей для машинного перевода. Стробирующая сеть обучается на предварительно обученной модели NMT ансамбля. 

Ансамбль локальных моделей имеет множество приложений в прикладных задачах. Работы \cite{Yumlu2003}, \cite{Cheung1995}\cite{Weigend2000} посвящены применению смеси экспертов в задачах прогнозирования временных
рядов. В работе \cite{article} предложен метод распознавания рукописных цифр. Метод распознания текстов при помощи ансамбля локальных моделей исследуется в работах \cite{Estabrooks2001}, а для распознавания речи --- в \cite{Mossavat2010}, \cite{Peng1996}. В работе \cite{Sminchisescu2007} исследуется смесь экспертов для задачи распознавания трехмерных движений человека.

\section{Постановка задачи построения ансамбля локальных моделей}
Пусть задано множество объектов $\Omega$. Будем считать, что множество объектов разбивается на $k$ непересекающихся подмножеств $\Omega_k$:
\[\Omega = \bigsqcup_{k=1}^K \Omega_k. \eqno(3.1)\]
Также задано $\Omega', |\Omega'| = N$ --- множество объектов для обучения,  являющееся подмножеством $\Omega$. Предполагается, что в $\Omega'$ представлены объекты из всех подмножеств $\Omega_k$:

\[\Omega' = \bigsqcup_{k=1}^K \Omega_k', \eqno(3.2)\]
где $\Omega_k' \subset \Omega_k$. Пусть задан вектор $\mathbf{y} \in \real^N$ --- вектор правильных ответов. Разбиение множества объектов $\Omega'$ на подмножества индуцирует разбиение вектора $\mathbf{y}$ на подвекторы $\mathbf{y}_k$. 

Для каждого объекта из $\Omega$ задано признаковое описание в соответствии с подмножеством, в котором объект находится. Это отображение $\mathcal{K}_k$ из множества объектов в пространство признаков:
\[\mathcal{K}_k: \Omega_k \rightarrow \real^{n_k}, k \in \overline{1, K}. \eqno(3.3)\]
В качестве общего пространства признаков будем рассматривать $\real^n = \real^{n_1} \times \dotsc \times \real^{n_K}$, полным признаковым описанием объекта является вектор $\mathbf{x} = [\mathbf{x}^1, \mathbf{x}^2, \dotsc, \mathbf{x}^K]$. Введем  выборку данных $\mathcal{D}$:
\[\mathcal{D} = \{(\mathbf{x}_i, y_i)~|~i \in \overline{1, N}\}, \eqno(3.4)\]
где $\mathbf{x}_i \in \real^n$ --- полное признаковое описание объекта из $\Omega'$, а $y_i \in \real$ --- значение целевой переменной, соответствующее этому объекту.
Для каждого подмножества используется своя локальная модель.\\
\begin{Definition}
\label{def:1}
Модель $\mathbf{g}_k$ называется локальной, если она аппроксимирует некоторую пару $(\Omega_k', \mathbf{y}_k)$.
\end{Definition}
В приведенном определении подразумевается, что локальная модель $\mathbf{g}_k$ использует только соответствующее признаковое описание $\mathbf{x}^{k}_i \in \real^{n_k}$ объекта --- подвектор вектора $\mathbf{x}_i$, соответствующий отображению $\mathcal{K}_k$. В данной работе локальный модели объединены в ансамбль локальных моделей.\\
\begin{Definition}
\label{def:2}
Ансамбль локальных моделей --- мультимодель, определяющая правдоподобие веса $\pi_k$ каждой локальной модели $\textbf{f}_k$ на признаковом описании объекта $\textbf{x}$.
\[\mathbf{f} = \sum\limits_{k=1}^K\pi_k\mathbf{g}_k(\mathbf{x}^{k},\mathbf{w}_k),\qquad \pi_k\left(\mathbf{x}, \mathbf{V}\right): \real^{n\times |\mathbf{V}|} \rightarrow [0,1], \qquad \sum\limits_{k=1}^K\pi_k\left(\mathbf{x}, \mathbf{V}\right) = 1, \eqno(3.5)\]
$\mathbf{f}$ --- мультимодель, $\mathbf{g}_k$ --- локальная модель, $\pi_k$ --- шлюзовая функция, $\mathbf{V}$--- параметры шлюзовой функции. 
\end{Definition}



В данной работе в качестве локальной модели $\mathbf{g}_k$ и шлюзовой функции $\bm{\pi}$ рассматриваются следующие функции:

\[\mathbf{g}_k(\mathbf{x}^{k}, \mathbf{w}_k) = \mathbf{w}_k^{\mathsf{T}}\mathbf{x}^{k}, \qquad \bm{\pi}\left(\mathbf{x}, \mathbf{V}\right) = \text{softmax}\left(\mathbf{V}_1^{\mathsf{T}}\sigma\left(\mathbf{V}_2^{\mathsf{T}}\mathbf{x}\right)\right), \eqno(3.6)\]
где $\mathbf{V} = \{\mathbf{V}_1, \mathbf{V}_2\}$ --- параметры шлюзовой функции, $\sigma(x)$ --- сигмоидная функция. Введем понятие расстояние между двумя объектами.\\
\begin{Definition}
\label{def:3}
Расстоянием между двумя объектами $\omega_1$ и $\omega_2$ из $\Omega$ называется число, равное расстоянию между векторами признаковых описаний этих объектов, и вычисляемое по формуле

\[\rho(\omega_1, \omega_2) = \left|\left|\mathcal{K}_1(\omega_1) - \mathcal{K}_1(\omega_2), \mathcal{K}_2(\omega_1) -  \mathcal{K}_2(\omega_2), \dotsc, \mathcal{K}_k(\omega_1) -  \mathcal{K}_k(\omega_2)\right|\right|_2. \eqno(3.7) \]
\end{Definition} 
Введенное расстояние на множестве объектов индунцирует расстояние  в пространстве моделей.\\
\begin{Definition}
\label{def:2}
Расстоянием между двумя моделями называется минимальное расстояние между двумя объектами, которые обрабатываются данными моделями
\[\rho(\mathbf{g}_i, \mathbf{g}_j) = \min\limits_{\omega_1 \in \Omega_i, \omega_2\in \Omega_j}{\rho(\omega_1, \omega_2)}. \eqno(3.8)\] 
\end{Definition}


Для нахождения оптимальных параметров мультимодели используется функция ошибки следующего вида:
\[\mathcal{L}\left(\textbf{V}, \textbf{W}\right) = \sum\limits_{(\textbf{x}, y) \in \mathcal{D}} \sum\limits_{k=1}^K\pi_k\left(\textbf{x}, \textbf{V}\right)\left(y - \textbf{w}_k^{\mathsf{T}}\textbf{x}^k\right)^2 + R\left(\mathbf{V}, \mathbf{W}\right), \eqno(3.9)\] 
где $\mathbf{W} = [\mathbf{w}_1, \mathbf{w}_2, \dotsc, \mathbf{w}_k]$ --- параметры локальных моделей, $R\left(\mathbf{V}, \mathbf{W}\right)$ --- регуляризация параметров.
\bibliographystyle{unsrt}
\bibliography{Islamov}

\end{document}
